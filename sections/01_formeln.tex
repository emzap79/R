\begin{longtabu}to\textwidth{l@{\hspace{3cm}}c@{\hspace{3.0em}}p{0.6\textwidth}}
    Erwartungswert: $E(X)$ &%{{{
    \parbox[c]{0.5\textwidth}{
        \begin{align*}
            \sum X\cdot P(X=x)\\
            E(X)=\frac{1}{n}\sum^{n}_{i=1} X=\bar X
        \end{align*}
    }
    &$\begin{dcases}
        \sum^{J}_{j=1}\lbrack x_jf(x_j)\rbrack\text{~~für X diskret}\\
        \int^{\infty}_{-\infty}\lbrack xf(x)dx\rbrack\text{~~für X indiskret}
    \end{dcases}$\\
    %}}}
    Bedingter EW: $E(Y|X=x)$ &$\sum y\cdot P(Y=y|X=x)$\\
    Bedingte Verteilung: $P(Y=y|X=x)$ &$\dfrac{P(Y=y, X=x)}{P(X=x)}$\\
    Varianz: $Var(X)$ & $E\left(X-E(X)\right)^2$ & $\; E(X^2)-E(X)^2$ \\
    Standardabweichung: $SD(X)$&$\sqrt{Var(X)}$&$\sqrt{E(X^2)-E(X)^2}$\\
    Kovarianz:%{{{
    $cov(X,Y)$ & $E \left[ (X-EX)(Y-EY)\right] $&$ \; E(XY) - E(X)E(Y)$\\
    && $cov(X,Y)=0$, falls X u. Y unabh.\\
    && $cov(X,X)=E[(X-\mu x)^2]=\sigma^2x$ \\%}}}
    Korrelation: $corr(X,Y)$  &%{{{
    $\dfrac{cov(X,Y)}{\sqrt{Var(X)}\sqrt{Var(Y)}}$ & $1\leq corr(X,Y)\leq
    1$\\\\%}}}
    Schiefe: g & $\dfrac{E(X-EX)^3}{SD(X)^3}$&$g < 0: \text{linksschief}$; $g > 0: \text{rechtsschief}$\\%{{{%}}}
    Kurtosis: k  %{{{
    & $\dfrac{E(X-EX)^4}{SD(X)^4}$ & Schwere der Flanken einer Verteilung\\
    &&$\Rightarrow k < 3: \text{schwach gewölbt}$\\
&&$\Rightarrow k > 3: \text{stark gewölbt}$\\%}}}
    Koeffizient der Bestimmtheit: $R^2$%{{{
    \label{formula:R2}
    &$\dfrac{ESS}{TSS}$
    &$R^2$ ist der Anteil \emph{erklärter} Variation von $Y_i$ an der Gesamtregression ($\;\hat{=}\;1-\frac{SSR}{TSS}$)\\%}}}
    Das Angepasste $R^2$: $\bar{R}^2$&$1-\left( \frac{n-1}{n-k-1}\right)\frac{SSR}{TSS}$&Das \frqq adjustierte\flqq{} $R^2$ wird
    kleiner, da der Effekt der Regressoren auf das Modell herausgerechnet wird.\\
    Explained Sum of Squares: $ESS$
    &$\sum^{n}_{i=1} \left( \hat{Y}_i - \bar{\hat{Y}}\right)^2$&Die Summe der
    durch das Modell erklärten quadrierten Abweichung.\\
    Residual Sum of Squares: $SSR$
    & $\sum^{n}_{i=1} \hat{u}_i^2$&Summe der quadrierten Residuen.\\
    Total Sum of Squares: $TSS$
    &$\sum^{n}_{i=1} \left( Y_i - \bar Y \right)^2$&$\hat{=}\;ESS + SSR$\\
    Standardfehler der Regression: $SER$%{{{
    \label{formula:SER}&
    $\sqrt{\frac{1}{n-2}\sum \hat u_i^2 }$~~, mit $\sum\hat u_i^2 = SSR$&
    $\varnothing$-Größe der OLS Residuen\\
    &&$\Rightarrow SER$: Geschätzte SA von $u$\\
&&$\Rightarrow RMSE$: Geschätzte Standardabw. von $u$, ohne Korrektur der
    Freiheitsgrade.\\%}}}
    {\parbox[t]{0.5\textwidth}{
            F-Statistik bei Homoskedastie\\
            \begin{itemize}
                \item $k_{\text{unrestr}}=$ {\small Anzahl der Regressoren in der unrestr. Regression.}
                \item $q=${\small Anzahl der Restriktionen unter der Nullhypothese}
            \end{itemize}
        }
    }
    & $\frac{R^2_{\text{unrestr}} - R^2_{\text{restr}}/q} {1-R^2_{\text{unrestr}}/n-k_{\text{unrestr}}-1}$
    & Wenn die Fehler homoskedastisch sind, folgt die F-Statistik bei
    Homoskedastie einer $\chi^2_q/q$ -Verteilung in großen Stichproben. Die
    F-Statistik bei Homoskedastie ist historisch wichtig, aber nicht valide bei
    Heteroskedastie.\\
    &
\end{longtabu}
