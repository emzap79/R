% subsect %%% binare_variable %{{{
\setcounter{equation}{0}
Übung 4.5: Bestimmung des $R^2$ sowie $SER$ (siehe \nameref{formula:SER})
einer binären Regression, bei Umkehrung der Dummy-Variable. Regression vorher
(1) und jetzt (2):

\begin{align}
	& \widehat{\text{Lohn}_i} = \beta_0 + \beta_1 \times \text{Mann}_i + u_i \label{eq:lohn_mann}
\end{align}

\begin{align}
	\begin{split}
		\widehat{\text{Lohn}_i} &= \gamma_0 + \gamma_1 \times \text{Frau}_i + v_i \label{eq:lohn_frau}\\
		& = \gamma_0 + \gamma_1 (1-\text{Mann}_i) + v_i\\
		& = \underbrace{\gamma_0 + \gamma_1}_{\hat{=}\beta_0} \underbrace{- \gamma_1}_{\hat{=}\beta_1} \times\text{Mann}_i + v_i
	\end{split}
\end{align}\\

hieraus folgt:

\begin{align}
	\begin{cases}
		\beta_{0}&=\gamma_0 + \gamma_1\\
	\textcolor{blue}{\beta_1}&\textcolor{blue}{=-\gamma_1}
	\end{cases}
	\text{~und~}
	\begin{cases}
		\gamma_0&=\beta_0 + \textcolor{blue}{\beta_1}\\
	\gamma_1&= -\beta_1
	\end{cases}
\end{align}

Wegen Beziehung geschätzter Koeffizienten gilt: $\hat{u}_i=\hat{v}_i$
und es ergibt sich\ldots

\begin{align*}
	\begin{rcases}
		SSR&=\sum^{n}_{i=1} \hat{u_i}^2\\
		SER&=\sqrt{\frac{SSR}{n-1}} \\
	\end{rcases}
	\text{bleiben unverändert, und somit auch $R^2$ gleich.}
	\marginnote{Beachte: $R^2\mDef 1- \frac{SSR}{TSS}$}
\end{align*}


% subsect %%% binare_variable (end) %}}}
